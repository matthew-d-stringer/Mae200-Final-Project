\documentclass{article}
\usepackage[utf8]{inputenc}
\usepackage{graphicx}
\usepackage{fancyhdr}
\usepackage{amsfonts}
\usepackage{amsmath}
\usepackage{amsthm}
\usepackage{amssymb}
\usepackage{enumitem}
\usepackage{listings}
\usepackage[
    letterpaper,
    margin=0.4in,
    tmargin=0.7in,
    bmargin=0.7in,
    headsep=12pt,
    footskip=12pt
]{geometry}

\theoremstyle{definition}
\newtheorem*{ques}{Question}

\renewcommand{\Re}{\operatorname{Re}}
\newcommand{\diam}{\operatorname{diam}}
\newcommand{\RR}{\mathbb{R}}
\newcommand{\trace}{\operatorname{trace}}

% <Laplace Transform>
\usepackage{mathrsfs}
\newsavebox\foobox
\newlength{\foodim}
\newcommand{\slantbox}[2][0]{\mbox{%
        \sbox{\foobox}{#2}%
        \foodim=#1\wd\foobox
        \hskip \wd\foobox
        \hskip -0.5\foodim
        \pdfsave
        \pdfsetmatrix{1 0 #1 1}%
        \llap{\usebox{\foobox}}%
        \pdfrestore
        \hskip 0.5\foodim
}}
\def\Laplace{\slantbox[-.45]{$\mathscr{L}$}}
% </Laplace Transform>

\pagestyle{fancy}
\fancyhf{}
\rhead{MAE 200}
\lhead{Matthew Stringer}
\chead{Final project}
\cfoot{\thepage}

\title{MAE 200 Final project}
\author{Matthew Stringer}
\date{}

\renewcommand{\ques}[1]{\section*{Question #1.}}

\begin{document}
    \maketitle
    \section*{Step 1}
    
    \section*{Step 2}
    
    \section*{Step 3}

    \section*{Step 4: State Estimation based on $\alpha$-horizon}
    I began with constructing my model of my system based on the linearized
    model around equation 22.34 of Numerical Renaissance. 
    This resulted in the following code:
    \begin{verbatim}
        E = [
            mc+m1+m2  -m1*l1       -m2*l1;
            -m1*l1   I1+m1*l1^2      0  ;
            -m2*l2    0          (I2 + m2*l2^2);
        ];

        E = [
            eye(3) zeros(3)
            zeros(3) E
        ];

        A_bar = [
            0   0       0
            0 m1*g*l2   0
            0   0      m2*g*l2
        ];

        A_bar = [ 
            zeros(3)  eye(3) 
            A_bar zeros(3)
        ];

        B_bar = [
            0
            0
            0
            1
            0
            0
        ];
    \end{verbatim}
    Since $E$ is invertible around $\vec q = \vec 0$, we can solve for 
    the $A$ and $B$ matrices from the standard form,
    \begin{eqnarray*}
        \dot q = A q + B u,
    \end{eqnarray*}
    by inverting the $E$ matrix. 
    Thus, we create the following code
    \begin{verbatim}
        A = inv(E)*A_bar;
        B = inv(E)*B_bar;

        C = eye(3, 6);
        D = 0;

        sys = ss(A,B,C,D);
    \end{verbatim}
    After running this code, we are left with the following system
    \begin{verbatim}
        sys =
            A = 
                    x1      x2      x3      x4      x5      x6
            x1       0       0       0       1       0       0
            x2       0       0       0       0       1       0
            x3       0       0       0       0       0       1
            x4       0   0.491   0.982       0       0       0
            x5       0   5.175  0.9428       0       0       0
            x6       0  0.9428    20.7       0       0       0
            
            B = 
                    u1
            x1       0
            x2       0
            x3       0
            x4  0.1044
            x5  0.1002
            x6  0.2004
            
            C = 
                x1  x2  x3  x4  x5  x6
            y1   1   0   0   0   0   0
            y2   0   1   0   0   0   0
            y3   0   0   1   0   0   0
            
            D = 
                u1
            y1   0
            y2   0
            y3   0
            
            Continuous-time state-space model.
    \end{verbatim}

    \section*{Step 5}
\end{document}