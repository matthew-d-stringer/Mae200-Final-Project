\documentclass{article}
\usepackage[utf8]{inputenc}
\usepackage{graphicx}
\usepackage{hyperref}
\usepackage{fancyhdr}
\usepackage{amsfonts}
\usepackage{amsmath}
\usepackage{amsthm}
\usepackage{amssymb}
\usepackage{enumitem}
\usepackage{listings}
\usepackage[
    letterpaper,
    margin=0.4in,
    tmargin=0.7in,
    bmargin=0.7in,
    headsep=12pt,
    footskip=12pt
]{geometry}

\theoremstyle{definition}
\newtheorem*{ques}{Question}

\renewcommand{\Re}{\operatorname{Re}}
\newcommand{\diam}{\operatorname{diam}}
\newcommand{\RR}{\mathbb{R}}
\newcommand{\trace}{\operatorname{trace}}

% <Laplace Transform>
\usepackage{mathrsfs}
\newsavebox\foobox
\newlength{\foodim}
\newcommand{\slantbox}[2][0]{\mbox{%
        \sbox{\foobox}{#2}%
        \foodim=#1\wd\foobox
        \hskip \wd\foobox
        \hskip -0.5\foodim
        \pdfsave
        \pdfsetmatrix{1 0 #1 1}%
        \llap{\usebox{\foobox}}%
        \pdfrestore
        \hskip 0.5\foodim
}}
\def\Laplace{\slantbox[-.45]{$\mathscr{L}$}}
% </Laplace Transform>

\pagestyle{fancy}
\fancyhf{}
\rhead{MAE 200}
\lhead{Matthew Stringer}
\chead{Final project}
\cfoot{\thepage}

\title{MAE 200 Final project}
\author{Matthew Stringer}
\date{}

\renewcommand{\ques}[1]{\section*{Question #1.}}

\begin{document}
    \maketitle
    \noindent Based on code from 
    \href{https://github.com/matthew-d-stringer/Mae200-Final-Project}{github.com/matthew-d-stringer/Mae200-Final-Project}

    \section*{Step 1: Computing $u(t)$ on $t \in [0,T]$}
    For step 1, we must use adjoint-based optimization in order to compute
    the optimal input $u_k$. To do this, we use the provided code from 
    Numerical Renaissance to calculate a more optimal input given a previous
    "guess" input. In order to get a satisfactory input, 2 things were done:
    First, an initial optimal input was generated given no input at all.
    Second, several more inputs were generated based on the previous computed
    optimal input.
    This results in a more optimal input.
    While generating inputs, I found that it was easier to generate satisfactory
    system responses when given a larger time horizon of 5 seconds rather than
    the default 3 second time horizon.
    This program in contained in \texttt{Dual\_Pendulum\_Input.m}
    \begin{center}
        \includegraphics*[width=4in]{Matlab Code/Step1_final_plot.png}
    \end{center}

    \section*{Step 2: State Estimation on $[0,T]$ based on noisy measurements}
    Depending on the covariance matrix of system noise and process noise, you can
    define $Q$ and $R$ weighting matrices. Then plugging these into equation 22.30
    of Numerical Renaissance, you can find the steady state value of the $P$ 
    matrix that enables you answer to calculate an ideal $L$ matrix. Using these
    values we can calculate the value of $L$ with
    \begin{equation*}
        L = -PC^H Q_2^{-1}.
    \end{equation*} 
    
    \section*{Step 3: Feedback Control}
    Using the Algebraic Ricatti Equation we can march backwards to determine the
    $X$ from equation 22.13a of Numerical Renaissance. By calculating this $X$ 
    matrix we can determine an optimal $K$ matrix. This is done by solving the
    equation,
    \begin{equation*}
        K = -R^{-1} B^H X.
    \end{equation*}

    \section*{Step 4: State Estimation based on $\alpha$-horizon}
    I began with constructing my model of my system based on the linearized
    model around equation 22.34 of Numerical Renaissance. 

    Since $E$ is invertible around $\vec q = \vec 0$, we can solve for 
    the $A$ and $B$ matrices from the standard form,
    \begin{eqnarray*}
        \dot q = A q + B u,
    \end{eqnarray*}
    by inverting the $E$ matrix. 

    After inputting these matrices into matlab, we are left with the following system:
    \begin{verbatim}
        sys =
            A = 
                    x1      x2      x3      x4      x5      x6
            x1       0       0       0       1       0       0
            x2       0       0       0       0       1       0
            x3       0       0       0       0       0       1
            x4       0   0.491   0.982       0       0       0
            x5       0   5.175  0.9428       0       0       0
            x6       0  0.9428    20.7       0       0       0
            
            B = 
                    u1
            x1       0
            x2       0
            x3       0
            x4  0.1044
            x5  0.1002
            x6  0.2004
            
            C = 
                x1  x2  x3  x4  x5  x6
            y1   1   0   0   0   0   0
            y2   0   1   0   0   0   0
            y3   0   0   1   0   0   0
            
            D = 
                u1
            y1   0
            y2   0
            y3   0
            
            Continuous-time state-space model.
    \end{verbatim}
    Setting our $Q$ matrix to 1, and our $R$ matrix to 1,
    we create a Kalman filter using the \texttt{kalman} function.
    This results in an $L$ matrix 
    \begin{verbatim}
        L =
            0.4575    0.3394    0.3830
            0.3394    4.5172    0.2583
            0.3830    0.2583    9.0800
            0.2356    0.9622    1.9510
            0.8249   10.2934    1.8422
            1.7895    1.7995   41.3303
    \end{verbatim}

    \section*{Step 5: Optimal Control}
    Using the linearized model described in Step 5, it is possible
    to utilize MATLAB's \texttt{lqr} function.
    Since we primarily care about the positional states of our system,
    we set their weights to be twice as large as their derivatives.
    \begin{verbatim}
        Q = diag([1 1 1 0.5 0.5 0.5]);
    \end{verbatim}
    Since we only have one input, we can set its weight to 1.
    \begin{verbatim}
        R = 1;
    \end{verbatim}
    This results in the following $K$ matrix:
    \begin{verbatim}
        K =
            1.0000 -376.2629  680.7842    6.1759 -175.6396  154.7618
    \end{verbatim}
\end{document}